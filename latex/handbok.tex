\documentclass{article}
\usepackage[rounded]{syntax}
\usepackage[utf8]{inputenc}
\usepackage{hyperref}
\usepackage[T1, OT1]{fontenc}
\DeclareTextSymbolDefault{\dh}{T1}
\DeclareTextSymbolDefault{\TH}{T1}
\DeclareTextSymbolDefault{\th}{T1}

\usepackage{float}
\floatstyle{boxed}
\restylefloat{figure}

\begin{document}

\begin{center}
{\LARGE \textbf{Handbók}} ~\\
\textbf{19 Apríl 2014}
\end{center}

\clearpage
\paragraph{\Large Inngangur} ~\\

Þessi þýðandi er fyrir nafnlaust mál, sem er "subset" af \emph{Morpho}. Málið er eins uppset en
hefur mjög takmarkaða möguleika. Þýðandinn er skrifaður í \emph{python} forritunarmálinu. Keyrslu skáinn er \textbf{pmc.py} 
í rótinni af útgáfuni.
~\\
~\\
Allan kóða er hægt að sækja hér: \url{https://github.com/robotis/pmc}.

\clearpage
\paragraph{{\Large Notkun og uppsetning}} ~\\

Pakkin \href{http://www.dabeaz.com/ply/ply.html}{python-ply} þarf að vera uppsettur til að keyra pmc.py.

\paragraph{pmc.py} ~\\

\begin{verbatim}
usage: pmc.py [-h] [-v] [-V] [-g] [-S] [-L] [-B] [-m] [-q] target

Morpho (Subset) Compiler

positional arguments:
  target          morpho file

optional arguments:
  -h, --help      show this help message and exit
  -v, --version   show program's version number and exit
  -V, --verbose   Verbose mode

Debugging:
  -g, --debug     Debugging information
  -S, --scopemap  Scopemap
  -L, --lex-only  Run lexer on target
  -B, --bnf       Output BNF
  -m, --morpho    Run thru morpho (morpho command must be available)
  -q, --quite     Quite mode

\end{verbatim}

\clearpage
\paragraph{Dæmi um notkun} ~\\

Skráin "empty.m" inniheldur eitt tómt fall:
\begin{verbatim}
main = 
	fun()
	{
		
	};
\end{verbatim}

keyrð með pmc.py
\begin{verbatim}
$ python pmc.py empty.m
\end{verbatim}

Myndi skila eftirfarandi:

\begin{verbatim}
"out.mexe" = main in
{{
#"main[f0]" =
[
(MakeValR null)
];
}}
*
BASIS
;
\end{verbatim}

Þessi þýðandi setur alltaf \emph{module-block} í kringum úttakið, með nafninu "out.mexe". Einnig er \textbf{BASIS} alltaf tekinn inn. 
Þetta úttak er svo hægt að keyra gegnum \emph{morpho} til að fá keyrsluhæfa skrá:

\begin{verbatim}
$ python pmc.py empty.m > out.morpho
$ morpho -c out.morpho
Reused 1455 out of 2056 operations, 601 operation objects used.
Reuse ratio is 71%
$ morpho out
\end{verbatim}

Sem að sjálfsögðu myndi ekkert gera vegna þess að fallið er tómt.

\paragraph{Einingarpróf} ~\\

Hægt er að keyra einingarpróf með:
\begin{verbatim}
$ python test/test_files.py
\end{verbatim}

\clearpage
\paragraph{{\Large Málfræði}} ~\\

Málið er \emph{subset} af Morpho, svo þeir hlutar sem eru notaðir eru eins upp bygðir.

\paragraph{Lykilorð} ~\\
\begin{table}[h!]
	\begin{center}
	\begin{tabular}{ l c r }
	  else & false & fun \\     
	  if   & null  & return \\   
	  true & var \\   
	  \hline
	\end{tabular}
	\caption{Lykilorð (keywords)}
	\end{center}
\end{table}

\clearpage
\paragraph{Virkjar} ~\\

\begin{figure}[h!]
	\begin{syntdiag}
	\begin{stack}
		'\&\&' \\ '||' \\ '!=' \\
		'\textless=' \\ '\textgreater=' \\
		'\textless' \\ '\textgreater' \\
		'+' \\ '-' \\ '*' \\ '/' \\ '\%'
	\end{stack}
	\end{syntdiag}
	\caption{{\textless}oper{\textgreater}}
\end{figure}

\begin{table}[h!]
	\begin{center}
	\begin{tabular}{ l c r }
	Operator & Predecence \\
	\hline
	'/', '*', '\%' & 1 \\
	'+', '-' & 2 \\
	'\textless', '\textgreater', '=' & 3 \\
	't_and_op', 't_or_op' & 4 \\
	\end{tabular}
	\caption{Operator predecence}
	\end{center}
\end{table}

\paragraph{Dæmi um virkni:} ~\\

\begin{table}[h!]
	\begin{center}
	\begin{tabular}{ l c r }
	1 + 2 + 3 & og & (1 + (2 + 3)) \\
	1 + 2 * 3 & og & (1 + (2 * 3)) \\
	1 * 2 + 3 & og & ((1 * 2) + 3) \\
	\end{tabular}
	\caption{Operator examples}
	\end{center}
\end{table}

\clearpage
\paragraph{Fastar} ~\\

Fastar í málinu eru tölur, fleititölur og breytur. Einnig eru sérstakir fastar fyrir \emph{true}, \emph{false} og \emph{null}.
Tölur og fleititölur geta verið neikvæðar. Ekki er unnið með strengi eða stafi.

\begin{figure}[h!]
	\begin{syntdiag}
	\begin{stack}
		<t_name> \\ <t_int> \\ <t_float> \\ <t_null> \\ <t_false> \\ <t_true>
	\end{stack}
	\end{syntdiag}
	\caption{{\textless}literal{\textgreater}}
\end{figure}

\begin{figure}[h!]
	\begin{syntdiag}
	\begin{stack} \\
		'-' 
	\end{stack}
	\begin{rep}
		<0-9>
	\end{rep}
	\end{syntdiag}
	\caption{{\textless}t_int{\textgreater}}
\end{figure}

\begin{figure}[h!]
	\begin{syntdiag}
	\begin{stack} \\
		'-' 
	\end{stack}
	\begin{rep}
		<0-9>
	\end{rep}
	'.'
	\begin{rep}
		<0-9>
	\end{rep}
	\end{syntdiag}
	\caption{{\textless}t_float{\textgreater}}
\end{figure}

\paragraph{Sértákn} ~\\

'\{', '\}', '(', '')', ',' og ';' er tákn sem hafa merkingu.

\paragraph{Skjölun} ~\\

Öll \emph{comment} byrja á ';;;' fyrir stakar línur, og '\{;;;' til ';;;\}' fyrir margar línur.

\clearpage
\paragraph{{\Large Mállýsing}}

\setlength{\grammarparsep}{20pt plus 1pt minus 1pt} % increase separation between rules
\setlength{\grammarindent}{12em} % increase separation between LHS/RHS 

\begin{grammar}

<S'> ::= <program>

<program> ::= <program> <unit>
\alt <unit>

<unit> ::= <t\_name> '=' <fundecl> ';'

<statement> ::= <fundecl>
\alt <vardecl>
\alt <assign\_expr>
\alt <expr>

<statement\_list> ::= <statement\_list> ';' <statement>
\alt <statement>
\alt <empty>

<vardecl> ::= <t\_var> <vdecl>

<vdecl> ::= <vdecl> ',' <assign\_expr>
\alt <vdecl> ',' <t\_name>
\alt <assign\_expr>
\alt <t\_name>

<fundecl> ::= <t\_fun> '(' <params> ')' <body>
\alt <t\_fun> '(' <empty> ')' <body>

<params> ::= <params> ',' <t\_name> | <t\_name>

<body> ::= '{' <statement\_list> ';' '}'
\alt '{' <statement\_list> '}'

<exprlist> ::= <exprlist> ',' <expr> | <expr>

<assign\_expr> ::= <t\_name> '=' <expr>

<expr> ::= <literal>
\alt <call\_expr>
\alt <expr> '-' <expr>
\alt <expr> '+' <expr>
\alt <expr> '/' <expr>
\alt <expr> '*' <expr>
\alt <expr> '\%' <expr>
\alt <expr> '\textless' <expr>
\alt <expr> '\textgreater' <expr>
\alt <expr> <eq\_op> <expr>
\alt <expr> <t\_and\_op> <expr>
\alt <expr> <t\_or\_op> <expr>
\alt '-' <expr>
\alt '(' <expr> ')'
\alt <t\_return> <expr>
\alt <if\_expr>

<if\_expr> ::= <t\_if> '(' <expr> ')' <body> <elseiflist> <t\_else> <body>
\alt <t\_if> '(' <expr> ')' <body> <elseiflist>

<elseiflist> ::= <elseiflist> <t\_elsif> '(' <expr> ')' <body>
\alt <t\_elsif> '(' <expr> ')' <body>
\alt <empty>

<call\_expr> ::= <t\_name> '(' <exprlist> ')'

<eq\_op> ::= '\textgreater' '='
\alt '\textless' '='
\alt '!' '='

<literal> ::= <t\_integer>
\alt <t\_float>
\alt <t\_false>
\alt <t\_true>
\alt <t\_null>
\alt <t\_name>

<empty> ::= <empty>

\end{grammar}

\clearpage
\paragraph{{\Large Merking málsins}}

\paragraph{Forrit} ~\\

Forrit er skilgreint sem eitt eða fleiri föll. Þar af er eitt fall sem verður að heita "main".

\begin{figure}[h!]
	\begin{syntdiag}
	\begin{rep} 
		<fundecl> \\ ';'
	\end{rep} 
	\end{syntdiag}
	\caption{{\textless}program{\textgreater}}
\end{figure}

\clearpage
\paragraph{Föll} ~\\

Föll eru alltaf skilgreind með nafni.

\begin{figure}[h!]
	\begin{syntdiag}
	<t_name> '=' 'fun' '('
	\begin{stack} \\
		\begin{rep} 
			<t_name> \\ ','
		\end{rep} 
	\end{stack} 
	')' <body>
	\end{syntdiag}
	\caption{{\textless}fundecl{\textgreater}}
\end{figure}

\begin{figure}[h!]
	\begin{syntdiag}
	'\{'
	\begin{stack} \\
		\begin{rep} 
			\begin{stack}
				<vardecl> \\
				<expr> \\
				<if_expr> \\
				<t_return> 
				\begin{stack} \\
					<expr>
				\end{stack} 
			\end{stack} 
			';'
		\end{rep} 
	\end{stack} 
	'\}'
	\end{syntdiag}
	\caption{{\textless}body{\textgreater}}
\end{figure}

\paragraph{Breytur} ~\\

Breytur mega ekki byrja á tölustaf en mega annars innihalda [a-z0-9_]. Breytur eru null ef ekki settar.

\begin{figure}[h!]
	\begin{syntdiag}
	'var' 
	\begin{rep}
		<t_name> 
		\begin{stack} \\
			'=' <expr>
		\end{stack} \\ ','
	\end{rep} 
	';'
	\end{syntdiag}
	\caption{{\textless}vardecl{\textgreater}}
\end{figure}

\clearpage
\paragraph{Segðir} ~\\

\begin{figure}[h!]
	\begin{syntdiag}
	\begin{stack}
		<expr> <oper> <expr> \\
		<call_expr> \\
		'(' <expr> ')' \\
		<literal>
	\end{stack} 
	\end{syntdiag}
	\caption{{\textless}expr{\textgreater}}
\end{figure}

\paragraph{Kallsegð} ~\\
\begin{figure}[h!]
	\begin{syntdiag}
	<t_name> '('
	\begin{stack} \\
		\begin{rep} 
			<expr> \\ ','
		\end{rep} 
	\end{stack} 
	')'
	\end{syntdiag}
	\caption{{\textless}call\_expr{\textgreater}}
\end{figure}

\paragraph{if-segð} ~\\
\begin{figure}[h!]
	\begin{syntdiag}
	<t_if> '(' <expr> ')' <body>
	\begin{stack} \\
		<t_else> <body>
	\end{stack} 
	\end{syntdiag}
	\caption{{\textless}if\_expr{\textgreater}}
\end{figure}

\end{document}
